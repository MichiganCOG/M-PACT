% last updated in April 2002 by Antje Endemann
% Based on CVPR 07 and LNCS, with modifications by DAF, AZ and elle, 2008 and AA, 2010, and CC, 2011; TT, 2014; AAS, 2016

\documentclass{llncs}
\usepackage{graphicx}
\usepackage{amsmath,amssymb} % define this before the line numbering.
\usepackage{ruler}
\usepackage{subcaption}
\captionsetup{compatibility=false}
\usepackage{xcolor}
\usepackage[width=122mm,left=12mm,paperwidth=146mm,height=193mm,top=12mm,paperheight=217mm]{geometry}
\usepackage{multirow}

\newcommand{\jason}[1]{\textcolor{orange}{\textbf{JASON: #1}}}
\newcommand{\madan}[1]{\textcolor{red}{#1}}


\begin{document}
% \renewcommand\thelinenumber{\color[rgb]{0.2,0.5,0.8}\normalfont\sffamily\scriptsize\arabic{linenumber}\color[rgb]{0,0,0}}
% \renewcommand\makeLineNumber {\hss\thelinenumber\ \hspace{6mm} \rlap{\hskip\textwidth\ \hspace{6.5mm}\thelinenumber}}
% \linenumbers
\pagestyle{headings}
\mainmatter
\def\ECCV18SubNumber{1816}  % Insert your submission number here

\title{TARP: Tensorflow-based Activity Recognition Platform} % Replace with your title

\titlerunning{TARP: Tensorflow-based Activity Recognition Platform}

\authorrunning{Eric Hofesmann, Madan Ravi Ganesh \and Jason J. Corso}

\author{Eric Hofesmann, Madan Ravi Ganesh \and Jason J. Corso}
\institute{University of Michigan}
\newcommand{\acro}{TARP}

\maketitle

\begin{abstract}
\keywords{}
\end{abstract}

\section{Introduction}
\label{sec:intro}

\section{Overview of \acro}
\label{sec:overview}

\begin{figure}[b!]
\centering
\includegraphics[width=0.8\textwidth]{images/overview.pdf}
\caption{Illustration of the two main components of \acro}
\label{fig:overview}
\end{figure}

\acro~comprises of two main components, 1) the data input block, and 2) the execution block, as shown in Fig.~\ref{fig:overview}. 
The pipeline contained within the data input block can be divided into three simple stages,
\begin{enumerate}
\item Read video data from disk
\item Extract the desired number of clips from a given video
\item Preprocess the frames of clips using a selected model's preprocessing strategy.
\end{enumerate}
The execution block houses all of the code required to setup, train, test as well as log the outputs of a chosen model.
This includes defining the layers that comprise the model, training the model up to a predetermined number of epochs, saving parameter values of a model are regular intervals and finally testing the performance of the trained model over a variety of recognition metrics.
The following sections provide an in depth discussion of the setup and structure of various components of the platform.

\section{Input Pipeline}
\label{sec:ippipeline}

\section{Execution Block}
\label{sec:execblock}
Training and testing form the two main tenants of the execution block. 
Fig.~\ref{fig:exec_block} illustrates this partition along with the largest submodules present within each part.
Details regarding the flow of data and processes between various submodules and the entire execution block are provided below.

\begin{figure}[t!]
\centering
\includegraphics[width=0.8\columnwidth]{images/exec_block.pdf}
\caption{Training and Testing functions are the two major phases within the execution block. Model definitions and checkpoint-based functions are part of both training and testing functions while metrics are calculated after models are tested.}
\label{fig:exec_block}
\end{figure}

\subsection{Training: Flowchart of processes}
\label{sec:training}

\subsection{Model Description Submodule}
\label{sec:modeldesc}

\subsection{Checkpoint Submodule}
\label{sec:checkpoint}

\subsection{Testing: Flowchart of processes}
\label{sec:testing}

\subsection{Metrics Submodule}
\label{sec:metrics}

\clearpage

\bibliographystyle{splncs}
\bibliography{egbib}

\end{document}

\grid
\grid
\grid
\grid
\grid


